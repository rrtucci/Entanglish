\documentclass[12pt]{article}%\usepackage{squashed_ents_lab}
\usepackage{epsfig}     %standard package. Note graphics<graphicx<epsfig
%\usepackage{color}     %standard package
%\usepackage[pdfmark,colorlinks]{hyperref} %standard package written by Sebastian Rahtz
%\usepackage{subeqn}    %allows equations 1a, 1b
%\usepackage{subfig}    %allows figures 1a, 1b
\usepackage{amsmath}
\usepackage{amssymb}
\usepackage{url}

\paperheight=11in
\topmargin=0in
\headheight=0in
\headsep=0in
\topskip=0in
\textheight=8.5in
\footskip=.5in

\paperwidth=8.5in
\oddsidemargin=.25in
\evensidemargin=.25in
\textwidth=6.0in
\parindent=.5in

\newtheorem{claim}{Claim}
\newcommand{\proof}[0]{{\bf proof:} }
\newcommand{\qed}[0]{\newline\noindent{\bf QED }}


\newcommand{\bra}[1]{\langle#1|}
\newcommand{\ket}[1]{|#1\rangle}
\newcommand{\av}[1]{\langle#1\rangle}
\newcommand{\pder}[2]{\frac{\partial#1}{\partial#2}}
\newcommand{\tr}[0]{{\rm tr }}
\newcommand{\beq}{\begin{equation}}
\newcommand{\eeq}{\end{equation}}
\newcommand{\bsub}{\begin{subequations}}
\newcommand{\esub}{\end{subequations}}
\newcommand{\beqa}{\begin{eqnarray}}
\newcommand{\eeqa}{\end{eqnarray}}
\newcommand{\rarrow}[0]{\rightarrow}
\newcommand{\larrow}[0]{\leftarrow}
\newcommand{\Rarrow}[0]{\Rightarrow}
\newcommand{\nRarrow}[0]{\nRightarrow}
\newcommand{\Larrow}[0]{\Leftarrow}
\newcommand{\nLarrow}[0]{\nLeftarrow}
\newcommand{\ul}[1]{\underline{#1}}
\newcommand{\ZZ}[0]{ {\mathbb{Z}}}
\newcommand{\RR}[0]{{ \mathbb{R}} }
\newcommand{\CC}[0]{{ \mathbb{C}} }
\newcommand{\ground}{{}_{\stackrel{\stackrel{\displaystyle{\bot}}{-}}{.}}}


\newcommand{\rva}[0]{{\ul{a}}}
\newcommand{\rvb}[0]{{\ul{b}}}
\newcommand{\rvc}[0]{{\ul{c}}}
\newcommand{\rvd}[0]{{\ul{d}}}
\newcommand{\rve}[0]{{\ul{e}}}
\newcommand{\rvf}[0]{{\ul{f}}}
\newcommand{\rvg}[0]{{\ul{g}}}
\newcommand{\rvh}[0]{{\ul{h}}}
\newcommand{\rvi}[0]{{\ul{i}}}
\newcommand{\rvj}[0]{{\ul{j}}}
\newcommand{\rvk}[0]{{\ul{k}}}
\newcommand{\rvl}[0]{{\ul{l}}}
\newcommand{\rvm}[0]{{\ul{m}}}
\newcommand{\rvn}[0]{{\ul{n}}}
\newcommand{\rvo}[0]{{\ul{o}}}
\newcommand{\rvp}[0]{{\ul{p}}}
\newcommand{\rvq}[0]{{\ul{q}}}
\newcommand{\rvr}[0]{{\ul{r}}}
\newcommand{\rvs}[0]{{\ul{s}}}
\newcommand{\rvt}[0]{{\ul{t}}}
\newcommand{\rvu}[0]{{\ul{u}}}
\newcommand{\rvv}[0]{{\ul{v}}}
\newcommand{\rvw}[0]{{\ul{w}}}
\newcommand{\rvx}[0]{{\ul{x}}}
\newcommand{\rvy}[0]{{\ul{y}}}
\newcommand{\rvz}[0]{{\ul{z}}}

\newcommand{\rvX}[0]{{\ul{X}}}
\newcommand{\rvY}[0]{{\ul{Y}}}
\newcommand{\rvZ}[0]{{\ul{Z}}}

\newcommand{\rvA}[0]{{\ul{A}}}
\newcommand{\rvB}[0]{{\ul{B}}}
\newcommand{\rvC}[0]{{\ul{C}}}
\newcommand{\rvD}[0]{{\ul{D}}}




\newcommand{\proj}[0]{{\rm proj}}
\newcommand{\diag}[0]{{\rm diag}}
\newcommand{\biprod}[2]{\left[\begin{array}{c}#1\\#2\end{array}\right]^\times}
\newcommand{\setofall}[2]{\{#1\}_{\forall #2}}

\newcommand{\cala}[0]{{\cal A}}
\newcommand{\calb}[0]{{\cal B}}
\newcommand{\calc}[0]{{\cal C}}
\newcommand{\cald}[0]{{\cal D}}
\newcommand{\cale}[0]{{\cal E}}
\newcommand{\calh}[0]{{\cal H}}
\newcommand{\cali}[0]{{\cal I}}
\newcommand{\call}[0]{{\cal L}}
\newcommand{\cals}[0]{{\cal S}}
\newcommand{\calt}[0]{{\cal T}}



\newcommand{\lam}[0]{\lambda}

\newcommand{\nev}[0]{{n_{\rm eva}}}
\newcommand{\nra}[0]{{n_{\rm ra}}}
\newcommand{\nr}[0]{{n_{\rm rows}}}
\newcommand{\nab}[0]{{n_{\rm ab}}}
\newcommand{\nbs}[0]{{n_{\rm bs}}}
\newcommand{\norm}[1]{\parallel #1 \parallel_2}
\newcommand{\rvalp}[0]{{\ul{\alpha}}}
\newcommand{\alp}[0]{{\alpha}}
\newcommand{\alpk}[0]{{\alpha^{k}}}
\newcommand{\brazo}{ \begin{array}{r}
 \bra{0^{n_0}}
 \\
 \bra{1^{n_1}}
 \end{array}
 }
 \newcommand{\ketzo}{ \begin{array}{l}
 \ket{0^{n_0}}
 \\
 \ket{1^{n_1}}
 \end{array}
 }
\newcommand{\darrow}[0]{{\downarrow}}

\begin{document}

\title{A New  Algorithm for Calculating\\
Squashed Entanglement and a \\Python Implementation Thereof}


\author{Robert R. Tucci\\
        tucci@ar-tiste.com}


\date{ \today}

\maketitle

\vskip2cm
\section*{Abstract}
In this paper, we propose an algorithm
for calculating the squashed entanglement
and the entanglement of formation
of
any pure or mixed quantum state.
Our algorithm is a generalization
of the Arimoto-Blahut algorithm
 of classical
information theory,
where it is used to calculate channel capacities.
The author of this paper
published
a previous paper proposing a previous algorithm
for calculating squashed and formation entanglements.
Even though that previous algorithm was
also based on the Arimoto-Blahut algorithm, the
algorithm presented in this paper is
substantially different and performs much better.
This paper is being released at the same time that
a Python software library called ``Entanglish"
is being released at GitHub as open source.
Entanglish implements the algorithm for
calculating squashed and formation entanglements proposed
in this paper and also several other algorithms related
to quantum entanglement.
In order to test
the values of squashed and formation
entanglements that our algorithm
yields, we discuss
a family of pure states of multiple qubits
that we call the symmetrized N-up states.
We give and prove an analytical formula
for the pure state entanglement
(= squashed and formation entanglements since they are pure states) of this
family of states.
This paper also proposes a method for doing density matrix perturbation theory that we call
  ``bootstrapped perturbation theory".








\newpage
\section{Introduction}


In this paper, we propose an algorithm
for calculating the squashed entanglement
(Ref.\cite{sq-wiki}) and entanglement of
formation(Ref.\cite{Wot}) of
any pure or mixed quantum state.
Previous workers (Refs.\cite{Aud, Git, Ryu}) have
used gradient descent methods to
calculate entanglement of formation.
Our algorithm, on the other hand, is a generalization
of the Arimoto-Blahut algorithm (Ref.\cite{ab-algo})
 of classical
information theory,
where it is used to calculate channel capacities.

This paper is being released at the same time that
a Python library called ``Entanglish"
is being released at GitHub as open source.
Entanglish implements the algorithm for
calculating squashed and formation entanglements proposed
in this paper and many other algorithms related
to quantum entanglement.

In order to test
the values of squashed
entanglement that our algorithm
yields, we discuss
a family of pure states of multiple qubits
that we call the symmetrized N-up states.
We give and prove an analytical formula
for the pure state entanglement
(= squashed and formation entanglements since they are pure states) of this
family of states. What we call symmetrized N-up states
is not a new concept; they are often called Dicke states in the
physics literature.


One method of executing our algorithm for finding the
squashed entanglement of a quantum state $\rho$ requires
finding the eigen-decompositions of
various Hermitian matrices
of the same dimension
as the matrix $\rho$.
If the dimension of $\rho$
is small, finding
those eigen-decompositions,
with acceptable
speed and accuracy,
is feasible
using standard matrix
digitalization methods.
But
for dimension of $\rho$
bigger than a certain
value, these standard methods
will not work adequately.
We present an algorithm
that we call ``bootstrapped perturbation
theory" which
can handle better
than the standard
matrix digitalization methods
those $\rho$'s
with a large dimension.

For a review of the history of squashed
entanglement, see Ref.\cite{sq-wiki}.
The property
of strong subadditivity
of quantum entropy (i.e., that
$S(\rvx:\rvy|\rvalp)\geq 0$)
which guarantees that
squashed entanglement is non-negative, was first proven by Lieb and Ruskai in Ref.\cite{subadd-proof}.
The definition of squashed entanglement was first
proposed
and some of its properties
were first discovered by Tucci in a series of six papers
Refs.\cite{Tuc99}-\cite{Tuc02}
during the years 1999 to 2002.\footnote{
In his original 6 papers, Tucci didn't give any particular name to
his entanglement measure. The
name ``squashed  entanglement" was given
to it later by Christandl and Winter in Ref.\cite{Chr03}.
Some people prefer to call it ``CMI entanglement". CMI,
pronounced ``see-me", stands for Conditional Mutual Information.}
One of those six papers, Ref.\cite{Tuc01a},
proposed a method
for calculating squashed
entanglement that is different to the one
proposed in this paper, but also takes inspiration
from the Arimoto-Blahut algorithm.
The algorithm for calculating squashed
entanglement proposed in this paper
is substantially better and performs much better than its
predecessor.
The first paper to study
squashed entanglement
post Tucci's work appears to be
Ref.\cite{Chr03}, by Christandl and Winter in 2003.
They proved that squashed entanglement
satisfies other interesting properties
not touched upon in Tucci's work.
The fact that the squashed entanglement
of $\rho$ is
``faithful", meaning that it
is zero iff $\rho$ is ``separable"
was conjectured in the work of Tucci
(see  Ref.\cite{Tuc00a}) and was later proven
more rigorously in Refs.\cite{sep-proof1} and \cite{sep-proof2}.



\section{Notation and Preliminaries}
In this section, we will discuss some notation
and standard definitions that are used throughout the paper.

Let $\theta(\cals)$ be the so called truth or indicator
function. It equals 1 if statement $\cals$
is true and zero otherwise.
For example, $\theta(x>1)$ equals 1 when $x>1$
and zero otherwise.
$\delta(a, b)=\delta^a_b=\theta(a=b)$
 will denote the Kronecker
delta function.

We will use $Z_I$ to
denote the set of integers contained in an interval $I$,
where $I$ is an interval of real numbers
whose left side can be either
open or closed, and the same for its
right side. For example, $Z_{[0, 4)} = \{0, 1, 2, 3\}$
and
$Z_{[0, 4]} = \{0, 1, 2, 3, 4\}$.

For any positive integers $n,k$ with $k\leq n$, let

\beq
n^{\darrow k} = n(n-1)(n-2)\ldots (n-k+1)
\;.
\eeq
For example, $n^{\darrow 3} = n(n-1)(n-2)$.
$n^{\darrow k}$
has some properties in common with $n^k$,
the $k$'th power of $n$. ``$n$ choose $k$",
i.e.,
the number of combinations of $n$
objects taken $k$ at
a time, can be expressed using $n^{\darrow k}$.
Indeed, one has


\beq
{n\choose k} = \frac{n!}{k!(n-k)!}
 = \frac{n^{\darrow k}}{k!}
\;.
\eeq

For any square matrix $M$,  let
$\norm{M} = \sqrt{\tr(MM^\dag)}$,
where $M^\dag$ is the Hermitian
conjugate of $M$. This is called the 2-norm
or Frobenius norm of $M$. If all the entries of $M$ are
put into a column vector $v$, this is just the magnitude
$\sqrt{v^\dag v}$ of the vector.

Given any Hermitian matrix $\rho$, one can express $\rho$ as
an eigen-decomposition
$\rho=UDU^\dag$, where $U$ is a unitary
matrix and $D=\diag(
\lam_0, \lam_1, \ldots \lam_{\nev-1})$ is
a diagonal
matrix that contains the eigenvalues $\lam_j$  of $\rho$
along its diagonal. Let
$\pi_1(\rho)$ represent
the matrix one obtains
from the eigen-decomposition of $\rho$
 if one replaces
each $\lam_j$ in the matrix $D$
by $\lam_j\theta(|\lam_j|>\epsilon)$ for some small number
$\epsilon>0$. Let $\pi_0(\rho)=1-\pi_1(\rho)$,
so in $\pi_0(\rho)$, each $\lam_j$
in $D$ is replaced
by $\lam_j\theta(|\lam_j|\leq \epsilon)$.
$\pi_1(\rho)$ (resp., $\pi_0(\rho)$)
are the projectors onto the vector
space of non-zero (resp., zero)
eigenvalues of $\rho$. One can define the so called ``Penrose
pseudo-inverse" of $\rho$
 as:

\beq
{\rm pinv}(\rho)=
\pi_1(\rho)\rho^{-1} \pi_1(\rho)
\;.
\eeq
What this means is that, to arrive
at pinv$(\rho)$, one replaces
in the eigen-decomposition
of $\rho$, each
eigenvalue $\lam_j$ in $D$ by
$1/\lam_j$ if $|\lam_j|>\epsilon$
and by 0 otherwise.

In this paper, random variables will be
represented by underlined letters
 instead of by the more common
convention of using capital letters to represent them.
For example $P_{\rvx, \rvy}(x, y)=P(\rvx=x,\rvy=y)=P(x,y)$
will denote a joint distribution of the random variables
$\rvx$ and $\rvy$.

Let $\calh_\rvx$ denote a Hilbert (or vector) space
with an orthonormal basis $\{\ket{x}|x=0,1,2,\ldots, n_\rvx-1\}$.
We will refer to $n_\rvx$ as the size (or number of
possible values, or number of possible states) of $\rvx$.
Let $\call(\calh_\rvx)$ denote the set of all linear operators
acting on $\calh_\rvx$.
Let $\calh_{\rvx, \rvy}$ denote the tensor product of $\calh_\rvx$
and $\calh_\rvy$. If $\rho_{\rvx, \rvy}\in \call(\calh_{\rvx, \rvy})$
is a density matrix, then we define its partial traces as
$\rho_\rvx=\tr_\rvy\rho_{\rvx, \rvy}\in\call(\calh_{\rvx})$
and
$\rho_\rvy=\tr_\rvx\rho_{\rvx, \rvy}\in\call(\calh_{\rvy})$.

Next, we review a few standard definitions
used in classical and quantum
information theory.

In
classical information theory, one defines
the entropy of a classical
probability distribution $P_\rvx(x)$ by
\beq
H(\rvx) = -\sum_x P(x)\log P(x)
\;,
\eeq
where the sum is over all values that $\rvx$
can assume.
Then one defines the conditional information (CI) by

\beqa
H(\rvx|\rvy) &=&\sum_{x, y}
P(x, y) \log \frac{ 1}{P(x|y)}
\\
&=&
H(\rvx, \rvy)
-
H(\rvy)
\;,
\eeqa
and the conditional mutual information (CMI) by

\beqa
H(\rvx:\rvy|\rvalp) &=&\sum_{x, y, \alp}
P(x, y, \alp) \log \frac{ P(x, y,| \alp)}{P(x|\alp)P(y|\alp)}
\\
&=&
-H(\rvx, \rvy| \rvalp)
+
H(\rvx| \rvalp)
+
H(\rvy| \rvalp)
\;.
\eeqa
(The mutual information (MI) is the CMI without the $\rvalp$).

In quantum information theory, one defines
the von Neumann entropy of a density matrix $\rho_\rvx$
by

\beq
S(\rvx) = S(\rho_\rvx) =
-\tr_\rvx [\rho_\rvx \log \rho_\rvx]
\;,
\eeq
where the rows (and columns) of $\rho_\rvx$
are labelled by the  values that $\rvx$ can assume.
Then one defines the conditional information (CI) by

\beq
S(\rvx|\rvy) = S(\rvx, \rvy) - S(\rvy)
\;,
\eeq
and the conditional mutual information (CMI) by

\beq
S(\rvx : \rvy | \rvalp)=
-S(\rvx, \rvy| \rvalp)
+
S(\rvx| \rvalp)
+
S(\rvy| \rvalp)
\;.
\eeq
(The mutual information (MI) is the CMI without the $\rvalp$).


A maximally entangled pure state with two parts $\rvx$ and $\rvy$
is defined as a pure state whose density matrix $\rho_{\rvx,\rvy}\in \call({\calh_{\rvx, \rvy}})$
satisfies

\beq
S(\rho_\rvx)=S(\rho_\rvy)= \log n_{min}
\;,
\eeq
where $n_{min}=\min(n_\rvx, n_\rvy)$ and $n_\rvx, n_\rvy$
are the sizes of spaces $\calh_\rvx, \calh_\rvy$, respectively.
Let

\beq
\ket{\psi^{\rm max-entan}_{\rvx, \rvy}}=
\sum_{n=0}^{n_{min}}\ket{n}_\rvx\ket{n}_\rvy
\;,
\eeq
and

\beq
\rho^{\rm max-entan}_{\rvx, \rvy}=
\ket{\psi^{\rm max-entan}_{\rvx, \rvy}}
\bra{\psi^{\rm max-entan}_{\rvx, \rvy}}
\;.
\eeq
Then it is easy to show that
state $\rho^{\rm max-entan}_{\rvx, \rvy}$
is a maximally entangled state
with parts $\rvx$ and $\rvy$.





\section{Squashed Entanglement}
\label{sec-squashed}
In this section, we will
propose an algorithm for
calculating the squashed entanglement
of any pure or mixed state.


Consider a bipartite system
consisting of two parts labelled
by the random variables $\rvx$ and $\rvy$,
and
described by a density matrix $\rho_{\rvx, \rvy}$.
The squashed entanglement of  such a system
is defined as

\beq
E_{\rvx, \rvy}(\rho_{\rvx, \rvy}) =
\frac{1}{2}
\min S(\rvx : \rvy|\rvalp)
\;.
\eeq
The min()---or infimum()
if one wishes to be more mathematically
precise\footnote{The author
of this paper is a physicist. He is
often lax in Bourbaki style rigor, as most physicists are wont to be,
with good reason.  Physicists often
 skip excessive rigor, because it's not the goal of physics,
 as it is of mathematics, where it is essential.}--is
over all density matrices $\rho_{\rvx, \rvy,\rvalp}$
such that $\tr_\rvalp \; \rho_{\rvx, \rvy,\rvalp}=
\rho_{\rvx, \rvy}$ with $\rho_{\rvx, \rvy}$ held fixed.
This definition of $E_{\rvx,\rvy}$
has many desirable properties. For example,


\begin{itemize}
\item
$E_{\rvx, \rvy} = E_{\rvy, \rvx}\geq 0$. This is true
because of the so called ``strong subadditivity
of quantum entropy", which states that
$S(\rvx : \rvy|\rvalp)\geq 0$
for all density matrices
$\rho_{\rvx,\rvy, \rvz}$.
See Ref.\cite{subadd-proof}
for the first ever proof of $S(\rvx : \rvy|\rvalp)\geq 0$.
The classical analogue $H(\rvx : \rvy|\rvalp)\geq 0$
was known to be true long before Ref.\cite{subadd-proof}.

\item
If $\rho_{\rvx, \rvy}$ is pure (which is true iff
$\tr( \rho_{\rvx, \rvy}^2 - \rho_{\rvx, \rvy}) = 0$), then, as
Tucci pointed out in his very first paper on squashed entanglement
(this was one of his original motivations
for defining squashed entanglement)\footnote{ The factor of 1/2 in
the definition of squashed entanglement
is necessary to make this true.
Contrary to what Ref.\cite{sep-proof2} claims,
this factor
of 1/2 appears in all of Tucci's
 papers about squashed entanglement,
from the very earliest.}

\beq
E_{\rvx, \rvy} = S(\rvx) = S(\rvy)
\;.
\eeq

\item
$E_{\rvy, \rvx}(\rho_{\rvx, \rvy})= 0$ iff
$\rho_{\rvx, \rvy}$
is conditionally separable,
by which we mean that $\rho_{\rvx, \rvy}$
can be expressed as\footnote{
It is more conventional in the literature to call
property Eq.(\ref{eq-stoch-sep})
simply ``separability" instead of conditional
separability, but I prefer to use the word
``separable" to refer to case that
$\rho_{\rvx,\rvy} = \rho_\rvx \rho_\rvy$.}

\beq
\rho_{\rvx,\rvy}= \sum_\alp w^\alp
\rho^\alp_\rvx \rho^\alp_\rvy
\;,
\label{eq-stoch-sep}
\eeq
where $w^\alp$ for all $\alp$
is a probability distribution,
and $\rho_\rvx^\alp, \rho_\rvy^\alp$
are density matrices.
This property is enunciated by some by saying that
``squashed entanglement is faithful".
In particular,
this implies
in the classical case when
$\rho_{\rvx,\rvy}$ is diagonal with diagonal
$P(x, y)$, that squashed entanglement is zero iff
$P(x,y)$ is expressible as

\beq
P(x,y)=\sum_{\alp}P(\alp)P(x|\alp)P(y|\alp)
\;.
\eeq
This last equation expresses conditional separabilty
in the classical case and is equivalent to the classical Bayesian
network $\rvx\leftarrow\rvalp\rightarrow\rvy$.\footnote{This fact, which can be enunciated
by saying that squashed
entanglement is faithful in the classical case, was
first pointed out in Tucci's original 6 papers on squashed entanglement
and was one of his original motivations for defining
squashed entanglement.}



\end{itemize}

We will assume that $\rho_{\rvx, \rvy, \rvalp}$
is of the form
\beq
\rho_{\rvx, \rvy, \rvalp} =
\sum_\alp w^\alp \ket{\alp}\bra{\alp} \rho_{\rvx, \rvy}^\alp
\;,
\eeq
where $w^\alp$ for $\alp\in Z_{[0, n_\rvalp)}$
is a probability distribution, where $\rho^\alp_{\rvx, \rvy}$
for each $\alp$ is a density matrix,
and where the states $\ket{\alp}$
for all $\alp$ are orthonormal.
Therefore,

\beq
\rho_{\rvx, \rvy} =
\tr_\rvalp\; \rho_{\rvx, \rvy, \rvalp} =
\sum_\alp
w^\alp \rho_{\rvx, \rvy}^\alp
\;.
\eeq

Define


\beq
E^\alp_{\rvx, \rvy}=\frac{1}{2}\tr_{\rvx, \rvy}\left\{
\rho^{\alp}_{\rvx,\rvy}[
\log \rho^{\alp}_{\rvx,\rvy}
-
\log \rho^{\alp}_{\rvx}
\rho^{\alp}_{\rvy}]
\right\}
\;
\eeq
so that the squashed entanglement can be expressed as


\beq
E_{\rvx, \rvy}=\min
\sum_\alp w^\alp
E^\alp_{\rvx, \rvy}
\;.
\eeq

It is also convenient to define

\beq
K^\alp_{\rvx, \rvy}=\rho^\alp_{\rvx, \rvy}w^\alp
\;.
\eeq
Note that

\beq
w^\alp=\tr_{\rvx, \rvy}K^\alp_{\rvx, \rvy}, \;\;
\rho^\alp_{\rvx, \rvy}=\frac{K^\alp_{\rvx, \rvy}}{\tr_{\rvx, \rvy}K^\alp_{\rvx, \rvy}}
\;.
\eeq
Hence, one can always translate back and forth
between the pair $(w^\alp, \rho^\alp_{\rvx, \rvy})$ and
$K^\alp_{\rvx, \rvy}$ so one can think of these two quantities as being
2 equivalent representations of the same thing.

Suppose $M^\alp_{\rvx, \rvy}\in \call(\calh_{\rvx, \rvy})$ for all $\alp\in Z_{[0, n_\rvalp)}$.
For $\rho_{\rvx, \rvy}\in\call(\calh_{\rvx, \rvy})$, define the function ${\tt regulate}_{\rho_{\rvx, \rvy}}()$
by


\beq
{\tt regulate}_{\rho_{\rvx, \rvy}}(M^\alp_{\rvx, \rvy}) = X M^\alp_{\rvx, \rvy} X^\dag
 \;,
\eeq
where

\beq
X= \sqrt{\rho_{\rvx, \rvy}}\;\;{\rm pinv}
\left(\sqrt{\sum_\alp M^\alp_{\rvx, \rvy}}\right)
\;.
\eeq
Note that if 

\beq 
\rho_{\rvx, \rvy}=\sum_\alp M^\alp_{\rvx, \rvy}
\;,
\eeq
then:

\beq
\rho_{\rvx, \rvy}= \sum_\alp {\tt regulate}_{\rho_{\rvx, \rvy}}(M^\alp_{\rvx, \rvy})
\;.
\eeq

Define the Lagrangian

\beq
{\cal L}=
\sum_\alp w^\alp
E^\alp_{\rvx, \rvy}
+ \mu(\sum_\alp w^\alp -1)
+\tr_{\rvx, \rvy}\left[
\frac{\lam_{\rvx, \rvy}}{2}(\sum_\alp w^\alp
\rho^\alp_{\rvx, \rvy}-\rho_{\rvx,\rvy})
\right]
\;.
\eeq
To calculate $E_{\rvx, \rvy}$, we will set to zero
the variation of $\call$ when we vary $w^\alp\in \RR$, $\rho^\alp_{\rvx,\rvy}
\in \call(\calh_{\rvx, \rvy})$
and the Lagrange multipliers $\mu\in \RR$, $\lam_{\rvx, \rvy}\in\call(\calh_{\rvx, \rvy})$.


To calculate the optimum $\rho^\alp_{\rvx,\rvy}$, note that

\beq
\delta{\cal L}=\left\{
\begin{array}{l}
\sum_\alp \frac{w^\alp}{2}\tr_{\rvx, \rvy}\left[\delta\rho^\alp_{\rvx,\rvy}
(\log \rho^\alp_{\rvx,\rvy}+1
-\log \rho^{\alp}_{\rvx}
\rho^{\alp}_{\rvy}
+\lam_{\rvx,\rvy})\right]\\
+\sum_\alp\delta w^\alp\left[
E^\alp_{\rvx,\rvy}
+\tr_{\rvx, \rvy}
\frac{\lam_{\rvx,\rvy}}{2}
\rho^\alp_{\rvx,\rvy}
+\mu\right]
\end{array}
\right.
\;.
\eeq
The two independent variations with respect to $\rho^\alp_{\rvx,\rvy}$
and $w^\alp$ are zero iff

\beq
\left\{
\begin{array}{l}
\forall \alp,
\Delta_{\rvx, \rvy}=-\lam_{\rvx, \rvy}-1
=
\log \rho^{\alp}_{\rvx,\rvy}
-
\log \rho^{\alp}_{\rvx}
\rho^{\alp}_{\rvy}
\\
\mu=\frac{1}{2}
\end{array}
\right.
\;.
\eeq

The above results suggest
the following recursion relation to go from
$K^{\alp}_{\rvx,\rvy}(n)$
to $K^{\alp}_{\rvx,\rvy}(n+1)$ for $n=0,1,2,\ldots$.

Begin by setting $K^{\alp}_{\rvx,\rvy}(n=0)$
in either the ``eigen" or ``eigen+" patterns discussed in
Appendix \ref{sec-recursion-init}.



Let
\beq
\av{x^\alp}_\rvalp = \sum_{\alp=0}^{n_\rvalp-1}w^\alp x^\alp
\;.
\eeq



For $n=0,1,2, ..$, given $K^{\alp}_{\rvx,\rvy}(n)$ for all $\alp$,
choose $K^{\alp}_{\rvx,\rvy}(n+1)$ for all $\alp$ as follows.
Set

\beq
\Delta^\alp_{\rvx, \rvy}(n)=
\log \rho^{\alp}_{\rvx,\rvy}(n)
-
\log \rho^{\alp}_{\rvx}(n)
\rho^{\alp}_{\rvy}(n)
\;,
\eeq

\beq
\Delta_{\rvx, \rvy}(n)=
\av{\Delta^\alp_{\rvx, \rvy}(n)}_\rvalp
\;.
\eeq
Let

\beq
\cala^\alp(n)=
\exp^-\left[
\Delta_{\rvx, \rvy}(n)
+
\log \rho^{\alp}_{\rvx}(n)
\rho^{\alp}_{\rvy}(n)
\right]w^\alp(n)
\;,
\eeq
where $\exp^-(H)$ for any Hermitian matrix $H$ is calculated by expressing
$H=UDU^\dag$ where $U$ is a unitary matrix and $D$ is real diagonal matrix, and
replacing each diagonal value
$d$ of $D$ by $\exp(\min(0, d))$. Finally, let


\beq
K^{\alp}_{\rvx,\rvy}(n+1)
=
{\tt regulate}_{\rho_{\rvx, \rvy}}\left[
\cala^\alp(n)\right]
\;,\label{eq-squashed-only1}
\eeq

\beq
E_{\rvx, \rvy}(n+1) =
\frac{1}{2}\tr_{\rvx, \rvy} \left[
\sum_\alp K^\alp _{\rvx, \rvy}(n)
\Delta^\alp_{\rvx, \rvy}(n)\right]
\rarrow \frac{1}{2}\tr_{\rvx, \rvy} \left[\rho_{\rvx, \rvy}\Delta_{\rvx, \rvy}(\infty)\right]
\;,
\eeq

\beq
err(n+1)=
\av{\|\Delta^\alp_{\rvx, \rvy}(n)-\Delta_{\rvx, \rvy}(n)\|}_\rvalp
\;.
\eeq


We claim but will not prove rigorously that

\beq
E_{\rvx, \rvy}(n)\rightarrow
E_{\rvx, \rvy}
\;,
\eeq

\beq
err(n)\rightarrow 0
\;
\eeq
as $n\rightarrow \infty$.

\section{Formation Entanglement}

In this section, we explain that the entanglement of formation
can be viewed as a special case of the squashed
entanglement, and the algorithm discussed in Section
\ref{sec-squashed} can be easily
adapted to calculate entanglement of formation.\footnote{This was all pointed out
in Ref.\cite{ab-algo}}

We've been calculating squashed entanglement from the formula

\beq
E_{\rvx, \rvy}(\rho_{\rvx, \rvy})=
\frac{1}{2}
{\rm min}\sum_\alp w^\alp [
S(\rho^\alp_{\rvx})
+S(\rho^\alp_{\rvy})
-S(\rho^\alp_{\rvx, \rvy})]
\;,
\eeq
where the minimum is over all
families of density matrices
$\{\rho^\alp_{\rvx, \rvy}|\forall \alp\}$
such that $\sum_\alp \rho^\alp_{\rvx, \rvy}=
\rho_{\rvx, \rvy}$. If we just add
the further constraint that
the states $\rho^\alp_{\rvx, \rvy}$ are pure states, then
$S(\rho^\alp_{\rvx, \rvy})=0$,
$S(\rho^\alp_{\rvx})=S(\rho^\alp_{\rvy})$,
and we get precisely the definition
of Entanglement of Formation for a pure or mixed state.

To adapt the algorithm of Section \ref{sec-squashed}
to calculate formation entanglement,
simply
replace Eq.(\ref{eq-squashed-only1}) by the following.
For each $\alp$, let $w^\alp(n)$ be the largest eigenvalue
and $\ket{\psi^\alp(n)}$ its corresponding
eigenvector of $
\cala^\alp(n)$.
Then define

\beq
K^\alp_{\rvx, \rvy}(n+1)={\tt regulate}_{\rho_{\rvx, \rvy}}\left[
w^\alp(n)\ket{\psi^\alp(n)}\bra{\psi^\alp(n)}\right]
\;.
\eeq








\section{Bootstrapped 2nd Order Perturbation Theory}

In this section, we
propose a method for calculating perturbatively
any function $f(\rho)$ of $\rho$
 expressible
as a power series of $\rho$, for any quantum density
matrix $\rho$. We are especially interested
in using this method
to evaluate any function  $f(\cdot)$
that arises in the calculation of
squashed entanglement for pure or mixed states.

First, let us review the standard theory,
as used in Quantum Mechanics, of
(stationary, time independent)
 perturbation theory for a Hamiltonian $H$.
 We will denote the eigen-system
 of $H$ by

\beq
{\rm esys}(H) = \{ (\lam_j, v_j)\}_{j=0, 1, ..., \nev-1}
\;,
\eeq
where $\lam_j$ are the
eigenvalues of $H$ and the column vectors $v_j$
are eigenvectors satisfying $Hv_j = \lam_jv_j$.
One can define a unitary matrix $U$
by stacking
side by side the column vectors $v_j$,
and a diagonal matrix $D$
by placing the eigenvalues of
$H$ along the diagonal of $D$. Then

\beq
U = [ v_0, v_1, \ldots, v_{\nev-1}],\;\;\;
D = {\rm diag}(\lam_0, \lam_1, \ldots, \lam_{\nev-1})
\;
\eeq
and

\beq
H = \sum_{j=0, 1, ..., \nev-1} \lam_j v_jv^\dag_j
= UDU^\dag
\;.
\eeq

Suppose
\beq H = H_0 + V
\;,
\eeq
where $H_0$ is a Hermitian matrix for which esys($H_0$)
is known. $H_0$ is called the unperturbed
Hamiltonian and $V$ is called
the perturbation.  Perturbation
Theory is a method for finding an
approximation, call it ${\rm esys}^\approx(H)$,
for the exact eigen-system esys$(H)$ of $H$.
Symbolically, perturbation theory
takes us from esys$(H_0)$
to ${\rm esys}^\approx(H)$:


\beq
{\rm esys}(H_0) \rarrow {\rm esys}^\approx(H)
\;.
\eeq


The terms of the approximation ${\rm esys}^\approx(H)$
are grouped by their order (power) of $V$.
Ref\cite{pert-wiki} is quite useful because
it lists all terms up to fifth order!
 For completeness, we
list them here next,
up to 3rd order for the eigenvalues
and up to 2nd order for the eigenvectors.
To switch from our notation to the notation
of Ref.\cite{pert-wiki}, let
$\lam_j \rarrow E_j$ , $v_j\rarrow \ket{j}$,
$v_j^\dag \rarrow \bra{j}$. Also let

\beq
V_{nm} = \bra{n^{(0)}} V \ket{m^{(0)}}
\;,
\eeq

\beq
E_{nm} = E^{(0)}_n - E^{(0)}_m
\;,
\eeq

\beq
E_n = E^{(0)}_n + E^{(1)}_n +  E^{(2)}_n +\ldots
\;
\eeq

\beq
\ket{n} = \ket{n^{(0)}} +
 \ket{n^{(1)}} +\ket{n^{(2)}} + \ldots
\;.
\eeq

The eigenvalue approximations
 up to 3rd order are:


\beqa
E^{(1)}_n &=& V_{nn}
\\
E^{(2)}_n &=& \sum_{k_2\neq n}\frac{|V_{nk_2}|^2}{E_{nk_2}}
\\
E^{(3)}_n &=&
\sum_{k_2\neq n, k_3\neq n}
\frac{V_{nk_3}V_{k_3k_2}V_{k_2n}}{E_{nk_2}E_{nk_3}}
-\sum_{k_3\neq n} V_{nn}\frac{|V_{nk_3}|^2}{E^2_{nk_3}}
\;.
\eeqa
The eigenvector approximations
up to 2nd order are:

\beqa
\ket{n^{(1)}} &=&
\sum_{k_1\neq n} \frac{V_{k_1n}}{E_{nk_1}}\ket{k^{(0)}_1}
\\
\ket{n^{(2)}}&=&
\sum_{k_1\neq n, k_2\neq n}
\frac{V_{k_1k_2}V_{k_2n}}{E_{nk_1}E_{nk_2}}\ket{k_1^{(0)}}
- \sum_{k_1\neq n}\left(
\frac{V_{nn}V_{k_1n}}{E^2_{nk_1}}\ket{k^{(0)}_1}
+ \frac{V_{nk_1}V_{k_1n}}{2E_{nk_1}^2}\ket{n^{(0)}}
\right)
\;.
\eeqa

The higher order terms in
these perturbative approximations
to the eigenvalues and eigenvectors
diverge
when  some eigenvalues are degenerate,
i.e., when at least one eigenvalue
has an eigenspace that is of bigger
dimension than 1. For those cases,
one does what is called
``degenerate perturbation theory".
One does a similarity transformation
(change of bases) in each degenerate
eigenspace so as to diagonalize $V$
within that eigenspace.
After that change of
basis, for any $n,m$ with $n\neq m$,
if $E_{nm}=0$ then $V_{nm}=0$ too.
So now instead of getting expressions
that have a finite numerator and a zero
denominator, the
only problematic expressions that we get are
of the type 0/0, which we set to zero.


Next we try to answer the question, how
can we apply this machinery of perturbation
theory for Hamiltonians  $H$ to do perturbation
theory of density matrices $\rho$.
Both $H$ and $\rho$ are Hermitian matrices,
which insures that
their eigen systems exist and the eigenvalues are real
numbers. In the case of $\rho$,
the eigenvalues are also guaranteed to be $\geq 0$.
Let us begin by expressing $\rho$
as a sum of an unperturbed density matrix $\rho_0$
and a perturbation $\delta \rho$.

\beq
\rho = \rho_0 + \delta \rho
\;.
\eeq
Then the machinery of perturbation
theory will allow us to go from
esys($\rho_0$) to ${\rm esys^\approx(\rho)}$.

\beq
{\rm esys}(\rho_0) \rarrow {\rm esys^\approx(\rho)}
\;.
\eeq
It remains for us to define what is $\rho_0$.
We want a $\rho_0$ that
is easy to calculate
from $\rho$,
and such that esys($\rho_0$) is also easy to
calculate, much easier
than calculating esys($\rho$).
Next we define
a $\rho_0$ that satisfies this desiderata, and
which is also a particularly natural
choice for studying entanglement.

Before defining $\rho_0$, let us
define certain terms describing
the shape and partial traces of $\rho$.
For this, we will use a vocabulary
very close to the one used in the
documentation of the Python library numpy.

We will use a fixed $\rho$ in two different shapes:
either as an $\nr\times\nr$ matrix,
or as a tensor with $2\nra$ axes (axes are the same as
tensor indices). So
\beq
\rho \in \CC^{\nr\times\nr}
\simeq \CC^{[n_0\times n_1 \times\cdots\times(\nra-1)]^2}
\;
\eeq
where

\begin{itemize}
\item row shape = $(n_0, n_1, \ldots, \nra-1)$
\item $\nr$ = number of rows = product(row shape)
\item $\nra$ = number of row axes = len(row shape)
\end{itemize}
In the language of quantum information,
one can call say that $\rho$ describes $\nra$
qudits where $d=n_j$ for the $j'$th qudit.

Consider a $\rho$
with $\nra = 4$ for definiteness.
Such a $\rho$ can be represented by all of the following
equivalent notations:
\beq
\rho= \rho(0, 1, 2, 3) =\rho_{\rvx_0, \rvx_1, \rvx_2, \rvx_3}=
\rho^{j_0, j_1, j_2, j_3; j'_0, j'_1, j'_2, j'_3}
=\rho^{J;K}
\;
\eeq

Partial traces
of a density matrix
are fundamental to a rigorous definition
of entanglement. They are analogous to
finding the marginals of a
joint distribution in classical
probability. As an example of how we define
and express them in this paper,
suppose we want to
take the trace of $\rho(0, 1, 2, 3)$
with respect to the axes $1, 3$.
Using the notation of
this paper, we might write:

\beqa
\rho(0,2) &=&\rho_{\rvx_0,\rvx_2}
\\&=&
\tr_{\rvx_1, \rvx_3}\rho_{\rvx_0, \rvx_1, \rvx_2, \rvx_3}
\\&=&\sum_{j_1, j'_1}\sum_{j_3, j'_3}
\delta(j_1, j'_1)
\delta(j_3, j_3') \rho^{j_0, j_1, j_2, j_3; j'_0, j'_1, j'_2, j'_3}
\;.
\eeqa

Suppose that
\beq
{\rm All\;Row\;Axes} = A = Z_{[0, \nra)}
\;.
\eeq
Hence

\beq
\rho= \rho(A)
\;.
\eeq
We define

\beq
\rho(j) = \tr_{A-\{j\}}
\;\;\rho
\;
\eeq
and

\beq
\rho_0 = \rho(0)\otimes \rho(1)
\otimes\ldots \otimes \rho(\nra-1)
\;.
\eeq
We will refer to the density matrix $\rho_0$ as
the separable approximation of $\rho$.

To find esys($\rho_0$),
we only need to find
the esys($\rho(j)$) for each qudit $j$.
In fact, suppose

\beq
\rho(j) = U(j)D(j)U^\dag(j)
\;,
\eeq
for $j\in Z_{[0, \nra)}$ where $U(j)$
is a unitary matrix and $D(j)$
is a diagonal matrix. Then it follows that

\beq
\rho_0 = UDU^\dag
\;,
\eeq
where

\beqa
U &=& U(0)\otimes U(1)\otimes\ldots\otimes U(\nra-1)
\\
D &=& D(0)\otimes D(1)\otimes\ldots\otimes D(\nra-1)
\;.
\eeqa
In the case that the
row shape
of $\rho$ is $(2,2, \ldots 2)$
with $\nra$ components, i.e, if all the qudits are qubits,
each $\rho(j)$
is a 2-dim matrix which can be diagonalized
trivially. So,
as we stipulated in our desiderata,
finding esys($\rho_0$)
is much easier than finding
esys($\rho$), especially if
the dimension $\nr$ of $\rho$ is large.


With this definition of $\rho_0$,
there is no reason that
$\delta \rho = \rho -\rho_0$
must satisfy
$\norm{\delta\rho}  << 1$.
But we would like that inequality to be true
to insure that the perturbation theory
converges quickly.
If we define

\beq
\delta'\rho = \frac{\delta\rho }{\nbs},
 \;
 \eeq
 for some integer $\nbs$, then,
 for large enough $\nbs$,
 $\norm{\delta'\rho} <<1$
 will be true. This suggests what we
 will refer to as
 ``bootstrap" perturbation theory.
 The idea is to define a chain of
 density matrices, where adjacent
density matrices in the chain are
 a distance
 $\delta'\rho<<1$ apart,
 even if the beginning and end
 of the chain are much farther apart.
 Let

 \beq
 \rho_j = \rho_0 + j \delta'\rho
 \;\;\;{\rm for} \;\;j=0, 1, \ldots, \nbs-1
 \;.
 \eeq
 Then one can apply 2nd order perturbation
 theory to
 bootstrap ourselves
from $\rho_0$ to $\rho$
by advancing a small distance
$\delta'\rho$ many times
until we cover the full distance
$\delta\rho$. Bootstrap perturbation
can be represented symbolically by


\beq
{\rm esys}(\rho_0)\rarrow
{\rm esys}^\approx(\rho_1)\rarrow\cdots\rarrow
{\rm esys}^\approx(\rho_{\nbs-1})=
{\rm esys}^\approx(\rho)
\;,
\eeq
where approximations
are indicated by a $\approx$
superscript.


Consider any function $f(\rho)$ of $\rho$
expressible as a power series of $\rho$.
$f(\rho)$ can be many things. For example,
${\rm log}(\rho)$, ${\rm exp}(\rho)$,
$\sqrt{\rho}$, ${\rm pinv}(\rho)$
$\rho\log\rho$, etc.
One has

\beqa
f(\rho) &=& f(UDU^\dag) \approx
f(U^\approx D^\approx U^{\approx\dag})\\
&\approx&
U^\approx f(D^\approx})U^{\approx\dag
\;,
\eeqa
where, as before, $\rho=UDU^\dag$,
$U$ is unitary, $D$ is diagonal, and
approximations
are indicated by a $\approx$
superscript.

In conclusion,  bootstrap perturbation
allows us to
calculate esys$(\rho_0)$
easily,
to bootstrap
esys$(\rho_0)$ to
${\rm esys}^\approx(\rho)$,
and to approximate $f(\rho)$ from that
${\rm esys}^\approx(\rho)$. Finding such an approximation
 of $f(\rho)$ was the goal that
we set out for ourselves at
the beginning of this section.



\appendix
\section{Appendix: Recursion Initialization}
\label{sec-recursion-init}

The purpose of this
appendix is to define the starting family
$\{K^{\alp}_{\rvx,\rvy}|\alp=0, 1, \ldots, n_\rvalp-1\}$
used by Entanglish as a starting point ($n=0$)
of the recursion relation
proposed in this paper for
calculating squashed and formation entanglement.

Defining $K^{\alp}_{\rvx,\rvy}(n=0)=
\frac{1}{n_\rvalp}\rho_{\rvx,\rvy}$ for $\alp=0, 1, \ldots, n_\rvalp-1$, does not work because
this is a stationary point of the recursion. Try it.

The
next obvious choice is as follows. We call this
choice ``eigen".
Let $n_{rows}$ be the number of rows of $\rho_{\rvx, \rvy}$.
For $n=0$, define $n_\rvalp=n_{rows}$ and

\beq
K^{\alp}_{\rvx,\rvy}(0)= \lam^\alp
\ket{\psi^\alp}\bra{\psi^\alp}
\;
\eeq
for all $\alp$. Here $\ket{\psi^\alp}$ for $\alp=0, 1, \ldots n_\rvalp-1$
are orthonormal states and eigenvectors of $\rho_{\rvx, \rvy}$ such that

\beq
\rho_{\rvx, \rvy}= \sum_\alp \lam^\alp
\ket{\psi^\alp}\bra{\psi^\alp}
\;.
\eeq

The space spanning
$\rho_{\rvx, \rvy}$ has $n_{rows}^2$ degrees of freedom (dofs)
(in fact,
all Hermitian matrices with $n_{rows}$ rows have $n_{rows}^2$ dofs)
 and ``eigen" only has
$n_{rows}$ dofs. The following choice,
which we call ``eigen+", has $n_{rows}^2$ dofs,
which is sufficient though not necessary.
Let $\lam_{min}$ be the smallest non-zero
eigenvalue of $\rho_{\rvx, \rvy}$, and let
$n_{\neq 0}$ be the number of non-zero eigenvalues of
$\rho_{\rvx, \rvy}$. Define

\beq
\epsilon = \frac{\lam_{min}}{2(n_{\neq 0}-1)}\theta(n_{\ne0}> 0)
\;,
\eeq
\beq
\epsilon'= e^{i\frac{\pi}{2}}\sqrt{\epsilon(1-\epsilon)}
\;.
\eeq
Suppose $j, k, p, q\in \{0, 1, 2, \ldots, n_{rows}-1\}$. Let

\beq
w_{j,j}=
(1-\epsilon)\theta(\lam_j\neq 0)
\;,
\eeq
\beq
(\rho_{\rvx, \rvy}^{j,j})_{p, q}=\theta(p=j, q=j)
\;.
\eeq
For $j<k$, let

\beq
w_{j,k}=
\epsilon \theta(\lam_j\neq 0, \lam_k \neq 0)
\;,
\eeq

\beq
(\rho_{\rvx, \rvy}^{j,k})_{p,q}=
\left\{
\begin{array}{l}
(1-\epsilon)\theta(p=k, q=k)\\
+\epsilon\theta(p=j, q=j)\\
+\epsilon'\theta(p=j, q=k-1)\\
+(\epsilon')^* \theta(p=k-1, q=j)
\end{array}
\right.
\;,
\eeq

\beq
(\rho_{\rvx, \rvy}^{k, j})_{p,q}=
\left\{
\begin{array}{l}
\epsilon\theta(p=k, q=k)\\
+(1-\epsilon)\theta(p=j, q=j)\\
-\epsilon'\theta(p=j, q=k-1)\\
-(\epsilon')^* \theta(p=k-1, q=j)
\end{array}
\right.
\;.
\eeq
With $\alp=(j,k)$ , this family of states $\rho_{\rvx, \rvy}^\alp$
and weights $w^\alp$ satisfies $w^\alp\geq0$,
 $\sum_\alp w^\alp=1$, $n_\rvalp=n_{rows}^2$,
  $\rho_{\rvx, \rvy}^\alp$ are pure states,
and $\sum_\alp \rho_{\rvx, \rvy}^\alp =
\rho_{\rvx, \rvy}$.

\section{Appendix: Symmetrized N-up States}

In order to test the numerical algorithms
presented in this paper for calculating
squashed entanglement, it is very useful
to have a family of quantum states for which
the squashed entanglement is known
analytically. In this appendix, we describe
one such family of quantum states,
what we call the symmetrized N-up states.


The symmetrized N-up states
 are pure states composed of
multiple qubits (so no qudits with $d\neq 2$ in them).
They are simply obtained by starting
with a state
that is a tensor product
of  $n_1$ states $\ket{1}$
and $n_0$ states $\ket{0}$
and applying a full symmetrization
operator to that.

Let Sym($n$) be the group
of all permutations of $n$ objects
(often called the Symmetric Group on $n$ letters).
Then we define the full symmetrization
operator $S_n$ by



\beq
S_n = \frac{1}{n!} \sum_{\sigma \in {\rm Sym(n)}}\sigma
\;.
\eeq
For example, for $n=3$, we have

\beq
S_3 = \frac{1}{3!}[
1 + \sigma_{0,1}
+ \sigma_{1,2}
+ \sigma_{0,2}
+ \sigma_{0,1}\sigma_{1,2}
+ \sigma_{1,2}\sigma_{0,1}
]
\;,
\eeq
where $\sigma_{\alp, \beta}$ is the operator
that swaps qubits $\alp$ and $\beta$.
A nice discussion of the properties and use of $S_n$
in Group Theory can be
found in Ref.\cite{cvit} by Cvitanovic.
Note that $S_n$ is ``idempotent" (it equals its square):


\beqa
S_n^2 &=& \left(\frac{1}{n!}\right)^2 \sum_{\sigma_1}\sigma_1
\sum_{\sigma_2}\sigma_2
\\
 &=&\frac{1}{n!} \sum_{\sigma}\sigma
 \\
 &=& S_n
 \;.
 \eeqa


Let
 \beq
 n = n_0 + n_1
 \;,
 \eeq
 where $n_0, n_1\in Z_{[0, \infty)}$.
Let

 \beq
 \ketzo = \underbrace{
 \ket{0}\otimes\ket{0}\ldots\ket{0}}
 _{n_0\; {\rm factors}}
 \otimes
 \underbrace{
 \ket{1}\otimes\ket{1}\ldots\ket{1}}
 _{n_1\; {\rm factors}}
 \;.
 \eeq
 We define  the ``symmetrized
 $n_1$-up state" as the following $n$ qubit state:



 \beq
 \ket{\psi^{n}_{n_1}} =
 \sqrt{n\choose n_1}S_n
\ketzo
 \;.
 \eeq
 The normalization of these states has
 been chosen so that they satisfy
 the following condition:

 \begin{claim}
 \beq
 \av{\psi^{n}_{n_1} | \psi^{n}_{n'_1}}
 = \delta(n_1, n'_1)
 \;.
 \eeq
 \end{claim}
 \proof


 \beqa
 \av{\psi^{n}_{n_1} | \psi^{n}_{n_1}}
 &=&
{n\choose n_1}
\brazo
S_n
\ketzo
 \\
 &=& 1
 \;.
 \eeqa
 \qed


 \begin{claim}
 \beq
 \brazo \ket{\psi^n_{n_1}}=
 \begin{array}{c}
  \bra{1^{n_1}}
 \\
 \bra{0^{n_0}}
 \end{array}
 \ket{\psi^n_{n_1}} =
 \frac{1}{\sqrt{n\choose n_1}}
 \;.
 \eeq
 \end{claim}
 \proof

 \beqa
 \brazo \ket{\psi^n_{n_1}} &=&
 \sqrt{n\choose n_1}
 \brazo S_n \ketzo
 \\
 &=&
  \frac{1}{\sqrt{n\choose n_1}}
 \;.
 \eeqa
 \qed

 \begin{claim}

 \beq
 \ket{\psi^n_{n_1}}=
 \sqrt{\frac{n_0}{n}}
 \begin{array}{l}
 \ket{0}
 \\
 \ket{\psi^{n-1}_{n_1}}
 \end{array}
 +
 \sqrt{\frac{n_1}{n}}
 \begin{array}{l}
 \ket{1}
 \\
 \ket{\psi^{n-1}_{n_1-1}}
 \end{array}
 \;.
 \label{eq-top-qubit-resolved}
 \eeq
 \end{claim}
 \proof

 By symmetry, there
 have to exist $a,b \in \CC$
 so that

 \beq
 \ket{\psi^n_{n_1}}=
 a
 \begin{array}{l}
 \ket{0}
 \\
 \ket{\psi^{n-1}_{n_1}}
 \end{array}
 +
 b
 \begin{array}{l}
 \ket{1}
 \\
 \ket{\psi^{n-1}_{n_1-1}}
 \end{array}
 \;.
 \eeq

 To find $a$ and $b$, one notes that

 \beq
 \brazo \ket{\psi^n_{n_1}}
 =
 a
 \begin{array}{r}
  \bra{0^{n_0-1}}
 \\
 \bra{1^{n_1}}
 \end{array}
  \ket{\psi^{n-1}_{n_1}}
 \;,
 \eeq
 so

 \beq
 a = \sqrt{\frac{{n-1\choose n_1}}{{n\choose n_1}}}
=
 \sqrt{\frac{n_0}{n}}
 \;.
 \eeq

Likewise, one notes that
 \beq
  \begin{array}{r}
  \bra{1^{n_1}}
 \\
 \bra{0^{n_0}}
 \end{array}
 \ket{\psi^n_{n_1}}
 =
 b
 \begin{array}{r}
  \bra{1^{n_1-1}}
 \\
 \bra{0^{n_0}}
 \end{array}
  \ket{\psi^{n-1}_{n_1-1}}
 \;,
 \eeq
 so

 \beq
 b = \sqrt{\frac{{n-1\choose n_1-1}}{{n\choose n_1}}}
=
 \sqrt{\frac{n_1}{n}}
 \;.
 \eeq
 \qed

Let  $\Gamma$ be a subset of
$ Z_{[0,n)}$ and let $\Gamma^c$ be
$Z_{[0,n)}-\Gamma$, i.e.,  the complement
 of
$\Gamma$ in
$ Z_{[0,n)}$.
We will denote the number of elements
(cardinality) of these two sets by
$\gamma = |\Gamma|$,
$\gamma^c = |\Gamma^c|$.
Let

\beq
\rho^n_{n_1}= \rho^n_{n_1}(Z_{[0,n)})
=
\ket{\psi^n_{n_1}}
\bra{\psi^n_{n_1}}
\;.
\eeq
We define $\rho^n_{n_1}(\Gamma)$ by

\beq
\rho^n_{n_1}(\Gamma) =
\tr_{\Gamma^c}
\rho^n_{n_1}
\;.\label{eq-rho-n-none}
\eeq
The squashed entanglement is defined
for this case as simply the von Neumann entropy
of the trace over $\Gamma^c$ of $\rho^n_{n_1}$:

\beq
E_{\Gamma, \Gamma^c} = S[\rho^n_{n_1}(\Gamma)]
\;.
\label{eq-s-entang}
\eeq
The goal of the rest of this appendix will
be to find an analytical expression for the
right hand side of Eq.(\ref{eq-s-entang}).

Because the states
$\ket{\psi^n_{n_1}}$ from which
$\rho^n_{n_1}(\Gamma)$  is built are fully symmetrized,
$\rho^n_{n_1}(\Gamma)$ is the same for all $\Gamma$
with a fixed cardinality.
Thus,
we will assume from here on, without loss of generality, that


\beq
\Gamma^c = Z_{[0, \gamma^c)},\;\;
\Gamma = Z_{[\gamma^c, n)}
\;.
\eeq
Thus,
in Eq.(\ref{eq-rho-n-none}), we will take a
trace over the first (top) $\gamma^c$
qudits of $\ket{\psi^n_{n_1}}
\bra{\psi^n_{n_1}}$.

Multiplying Eq.(\ref{eq-top-qubit-resolved})
times its Hermitian conjugate,
and taking the trace
of this product over the top qudit,
one finds that

\beq
\rho^{n}_{n_1}(\Gamma)=
\frac{n_0}{n}\rho^{n-1}_{n_1}(\Gamma)
+
\frac{n_1}{n}\rho^{n-1}_{n_1-1}(\Gamma)
\;.
\label{eq-recurse-rho}
\eeq

Next, we will apply Eq.(\ref{eq-recurse-rho})
recursively. Each recursion will
``get rid of" the trace over the current top qudit.
After $\gamma^c-1$ recursions, we will have
gotten rid of the trace of the top
$\gamma^c$ qudits of the original
state $\ket{\psi^n_{n_1}}
\bra{\psi^n_{n_1}}$.


Here is the result of doing 1
 recursion of Eq.(\ref{eq-recurse-rho}):

\beqa
\rho^{n}_{n_1}(\Gamma)&=&
\frac{n_0}{n}\rho^{n-1}_{n_1}(\Gamma)
+
\frac{n_1}{n}\rho^{n-1}_{n_1-1}(\Gamma)
\\
&=&
\frac{n_0}{n}
\left[
\frac{n_0-1}{n-1}\rho^{n-2}_{n_1}(\Gamma)
+
\frac{n_1}{n-1}\rho^{n-2}_{n_1-1}(\Gamma)
\right]
\\
&& +
\frac{n_1}{n}
\left[
\frac{n_0}{n-1}\rho^{n-2}_{n_1-1}(\Gamma)
+
\frac{n_1-1}{n-1}\rho^{n-2}_{n_1-2}(\Gamma)
\right]
\\
&=&
\frac{n_0^{\darrow 2}}{n^{\darrow 2}}
\rho^{n-2}_{n_1}(\Gamma)
+
\frac{2n_0 n_1}{n^{\darrow 2}}
\rho^{n-2}_{n_1-1}(\Gamma)
+
\frac{n_1^{\darrow 2}}{n^{\darrow 2}}
\rho^{n-2}_{n_1-2}(\Gamma)
\;.
\eeqa

Here is the result of doing 2 recursions
of Eq.(\ref{eq-recurse-rho}):

\beq
\rho^{n}_{n_1}(\Gamma)=
\frac{n_0^{\darrow 3}}{n^{\darrow 3}}
\rho^{n-3}_{n_1}(\Gamma)
+
3\frac{n_0^{\darrow 2}n_1}{n^{\darrow 3}}
\rho^{n-3}_{n_1-1}(\Gamma)
+
3\frac{n_0 n_1^{\darrow 2}}{n^{\darrow 3}}
\rho^{n-3}_{n_1-2}(\Gamma)
+
\frac{n_1^{\darrow 3}}{n^{\darrow 2}}
\rho^{n-3}_{n_1-3}(\Gamma)
\;.
\label{eq-gamma-c-tres}
\eeq
Note that in Eq.(\ref{eq-gamma-c-tres}),

\beq
{\rm coefficient\;of\;}
\rho^{n-3}_{n_1-k} =
\left(\frac{3!}{n^{\darrow 3}}\right)
\left(\frac{n_0^{\darrow k}}{k!}\right)
\left(\frac{n_1^{\darrow 3-k}}{(3-k)!}\right)
=
\frac{{n_0\choose k}{n_1 \choose 3-k}}{{n\choose 3}}
\;
\label{eq-coef-gamma-c-tres}
\eeq
for $k=0,1,2,3$.

We claim at this point
that it is possible
to prove by induction
that if one goes the full way and recurses
Eq.(\ref{eq-recurse-rho}) $\gamma^c-1$
times, then
Eq.(\ref{eq-coef-gamma-c-tres})
is still valid but with
3 replaced by $\gamma^c$.

The right hand side of Eq.(\ref{eq-coef-gamma-c-tres})
is easy to recognize as the hypergeometric
probability distribution. In fact, under the
replacements
\beqa
k&\leftrightarrow&k
\\
K&\leftrightarrow&n_0
\\
N&\leftrightarrow&n
\\
n&\leftrightarrow&\gamma^c
\;,
\eeqa
one gets
\beq
P(k| K, N, n) =
\frac{{K\choose k}{N-K \choose n-k}}{{N\choose n}}
\;,
\eeq
which is the definition of the
hypergeometric distribution in
the notation of Ref.\cite{wiki-hypergeo}.
$P(k)$ for $k \in Z_{[0, n_0]}$
 is bell shaped with mean at $\gamma^c\;\;\frac{n_0}{n}$.

Finally, note that
if $\gamma^c=3$, then in Eq.(\ref{eq-gamma-c-tres}),
the states
$\rho^{n-3}_{n_1-k}$ on the right hand side are
mutually orthogonal pure states.
Hence, in the appropriate
basis,  Eq.(\ref{eq-gamma-c-tres})
is a diagonal matrix.
The von Neumann entropy of
a diagonal density matrix is
just
the classical entropy of
the probability distribution
that appears on the diagonal
of that diagonal density matrix.



\begin{thebibliography}{99}
\bibitem{sq-wiki}
\url{https://en.wikipedia.org/wiki/Squashed_entanglement}
\bibitem{Wot}
William K. Wootters,
``Entanglement of Formation of an Arbitrary State of Two Qubits",
\url{https://arxiv.org/abs/quant-ph/9709029}

\bibitem{Aud}
K. Audenaert, F. Verstraete, B. De Moor,
``Variational Characterisations of Separability and Entanglement of Formation",
\url{https://arxiv.org/abs/quant-ph/0006128}
\bibitem{Git}
J. R. Gittings, A. J. Fisher,
``An efficient numerical method for calculating the entanglement of formation of arbitrary mixed quantum states of any dimension",
\url{https://arxiv.org/abs/quant-ph/0302018}
\bibitem{Ryu}
Seunghwa Ryu, Wei Cai, and Alfredo Caro,
``Quantum Entanglement of Formation between Qudits",
Physical Review A 77, 052312 (2008),
\url{http://micro.stanford.edu/~caiwei/papers/Ryu_Cai_Caro_Entangle_v11.pdf}
\bibitem{ab-algo}
Richard E. Blahut, {\it Principles and Practice of Information Theory} (1987, Addison-Wesley)
\bibitem{subadd-proof} Elliott H. Lieb, Mary Beth Ruskai, ``Proof of the Strong Subadditivity of Quantum-Mechanical Entropy", Journal of Mathematical Physics 14 (1973) 1938-1941.
\bibitem{Tuc99} Tucci, Robert R. (1999). ``Quantum Entanglement and Conditional Information Transmission".
    \url{https://arxiv.org/abs/quant-ph/9909041}
\bibitem{Tuc00a} Tucci, Robert R. (2000). ``Separability of Density Matrices and Conditional Information Transmission".
    \url{https://arxiv.org/abs/quant-ph/0005119}
\bibitem{Tuc00b} Tucci, Robert R. (2000). ``Entanglement of Formation and Conditional Information Transmission".
    \url{https://arxiv.org/abs/quant-ph/0010041}
\bibitem{Tuc01a} Tucci, Robert R. (2001). ``Relaxation Method for Calculating Quantum Entanglement".
    \url{https://arxiv.org/abs/quant-ph/0101123}
\bibitem{Tuc01b} Tucci, Robert R. (2001). ``Entanglement of Bell Mixtures of Two Qubits". \url{https://arxiv.org/abs/quant-ph/0103040}
\bibitem{Tuc02} Tucci, Robert R. (2002). ``Entanglement of Distillation and Conditional Mutual Information".
    \url{https://arxiv.org/abs/quant-ph/0202144}

\bibitem{Chr03} Matthias Christandl, Andreas Winter (2003). ````Squashed Entanglement": An Additive Entanglement Measure". Journal of Mathematical Physics. 45 (3): 829–840. \url{https://arxiv.org/abs/quant-ph/0308088}

\bibitem{sep-proof1}
 P. Hayden, R. Jozsa, D. Petz, and A. Winter.  ``Structure of states which satisfy strong subadditivity of
quantum entropy with equality",
Comm. Math. Phys., 246:359, 2004

\bibitem{sep-proof2}
 Brandao, F., Christandl, M., Yard, J. (September 2011). ``Faithful Squashed Entanglement". Communications in Mathematical Physics. 306 (3): 805–830.
 \url{https://arxiv.org/abs/1010.1750}

\bibitem{pert-wiki}
\url{https://en.wikipedia.org/wiki/Perturbation_theory_(quantum_mechanics)}

\bibitem{cvit}
P. Cvitanović, {\it Group Theory: Birdtracks, Lie's, and Exceptional Groups}, Princeton University Press, Princeton 2008, available online at
\url{http://birdtracks.eu/}

\bibitem{wiki-hypergeo}
\url{https://en.wikipedia.org/wiki/Hypergeometric\_distribution}


\end{thebibliography}




\end{document}
